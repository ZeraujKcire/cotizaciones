\documentclass[5pt,letter]{report}

% === PLANILLA === (((
\input{0.preambulo1}
\usepackage{textcomp}
% \decimalpoint
%------------Diseño de página --------------------------------
\usepackage[centering, letterpaper,margin=2cm,top=1cm, headsep=24pt, headheight=2cm,includehead, includefoot]{geometry}

%------------Colores----------------------

\definecolor{principal}{RGB}{0, 52, 101}
\definecolor{secundario}{RGB}{102, 102, 102}
\definecolor{terciario}{RGB}{198,198,187}

%--------------Hipervinculos en color, ligas-azul, archivos-magenta, url-azul----------------
\hypersetup{
    colorlinks=true,
    linkcolor=secundario,
    filecolor=magenta,      
    urlcolor=blue}

%-------------Encabezado---------------------
\pagestyle{fancy}
\fancyhf{}

% \lhead{
% \includegraphics[width=4cm]{\rutaImagenes/logo_cp_14_nl}
% }

%--------------Pie de página------------------
\fancypagestyle{plain}{
	\fancyhf{}
	\fancyhead[L]{
	\includegraphics[height=1.5cm]{imagenes/duran_tax_studio}
	}
	\fancyhead[R]{
	\includegraphics[height=1cm]{imagenes/GENERAL/LOGOS/logo_valuami_fondo_blanco}
	}
	\fancyfoot[L]{
		\begin{tabular}{b{3cm}}
		\includegraphics[width= 1.5cm, page = 1]{imagenes/GENERAL/LOGOS/qr}
		\end{tabular}
	}
	\fancyfoot[C]{
	\begin{center}
	\bfseries \footnotesize 
	\textcolor{principal}{VALUAMI\textregistered} \\
		\scriptsize \it 
		Amargura 50, Interior 7 y 8, Antigua Granada Parques de la Herradura. \\ 
		Huixquilucan, Estado de M\'exico. CP. 52786 \\ 
		Tel. 52 94 7680 / 55 89 96 34.
	\end{center}
	}
	\fancyfoot[R]{\footnotesize{\thepage\hspace{3pt}de \pageref{LastPage}}}
}

\pagestyle{plain}

%------------Títulos de seccion y subseccion--------------



\renewcommand \thechapter {\Roman{chapter}}
\renewcommand \thesection {\Roman{section}}
\renewcommand \thesubsection {\thesection.\arabic{subsection}}
\renewcommand \thesubsubsection {\thesection.\arabic{subsection}.\arabic{subsubsection}}

\titleformat{\section}[hang]{\color{gray}\huge\bfseries}{\thesection.}{1em}{} 

\chapterfont{\color{principal}}
\sectionfont{\color{principal}}
\subsectionfont{\color{secundario}}
\subsubsectionfont{\color{terciario}}

\titleformat{\chapter}[display]{\normalfont\Large\filcenter\sffamily}
{%\titlerule[1pt]%
\vspace{1pt}%
%\titlerule
\vspace{1pc}%
\LARGE\MakeUppercase{\chaptertitlename} \thechapter}
{1pt}
{%\titlerule
%\vspace{1pc}%
\Huge}
\titlespacing{\chapter}{0pt}{*-11}{*1.5}

%------------------Profundidad del índice---------------------------

\setcounter{tocdepth}{3}
\setcounter{secnumdepth}{3}




%------------------Marca de agua---------------

\backgroundsetup{angle=0, contents={\includegraphics[width=2cm]{imagenes/GENERAL/LOGOS/isotipo_valuami}},opacity=0.2, scale=5}


\input{0.parametros_cotizacion}
% )))

\begin{document}

% === Presentacion === (((
\begin{flushright}
	28 de Agosto de 2024.
\end{flushright}
SR. JAVIER VAQUEIRO \\[4mm]
Estimado Javier: \\[4mm]
% )))
 
% === Descripcion === (((
En las siguientes líneas encontrarás la propuesta de los servicios profesionales que Estudio 
Durán, S.C. (en adelante, el “Estudio”), estaría en posibilidad de prestarles, en particular, 
respecto a nuestra participación en el análisis de las implicaciones fiscales en México y en 
Estados Unidos de América respecto de cierta estructura de inversión que mantiene en el 
extranjero una persona física residente fiscal en México (en adelante, el “Cliente”), 
considerando los planes de inversión que tiene a futuro. La presente sustituye a nuestra carta 
previa de fecha 25 de julio de 2024. \\ 

A continuación, se describe el alcance, las actividades, el costo de los servicios solicitados y otros 
aspectos de colaboración y acuerdo entre el Estudio y el Cliente.  
% )))
 
\section*{Alcance de nuestros servicios profesionales.} % (((
Esta propuesta de servicios ha sido elaborada con base en la información y documentación que 
nos ha sido proporcionada en correos electrónicos y llamadas telefónicas sostenidas con 
anterioridad. Atendiendo a la misma, si de alguna manera no hemos interpretado completa o 
correctamente el sentido y relevancia de la información y documentación referida, 
agradeceremos que ello nos sea informado a la brevedad para determinar si nuestras premisas 
o comentarios deben ser ajustados o modificados. 
% )))
 
\section*{Antecedentes.} % (((
Conforme a la información proporcionada entendemos que cierta persona física residente fiscal 
en México constituyó un Trust conforme a la legislación de Texas en los Estados Unidos de 
América; el cual, a su vez, ha organizado una entidad Limited Partnership en términos de la 
legislación de Alberta en Canadá, la cual según se nos ha informado no es reconocida como 
contribuyente ni en Canadá ni en los Estados Unidos de América. \\ 

Si bien, la LP canadiense prácticamente no ha tenido operaciones desde su constitución, 
entendemos que se tiene pensado en un futuro próximo realizar operaciones de inversión en 
instrumentos de renta variable a través de instituciones en los Estados Unidos de América. 
Inclusive, como parte de los planes se tiene contemplado también realizar inversiones de renta 
fija a través de una entidad ubicada en una cuarta jurisdicción.  \\ 

Derivado de lo antes expuesto, fue requerida la presentación de una propuesta de nuestros 
servicios profesionales en relación con el análisis de las implicaciones fiscales en México y en 
Estados Unidos de América respecto de la estructura de inversión señalada y considerando sus 
planes de inversion.
% )))

\section*{Plan de Trabajo.} % (((

Con base en lo señalado anteriormente, consideramos que las actividades comprendidas en la 
prestación de nuestros servicios podrían realizarse en tres fases e identificarse de la siguiente 
manera:

\subsection*{\it Primera Fase:} % (((
\begin{enumerate}[I.]
\item 
Revisión y análisis de la información y documentación proporcionada por el Cliente 
respecto a las entidades y figuras que forman parte de la estructura, así como respecto 
de las modificaciones y transacciones (aún operativas) que se hayan realizado desde su 
constitución y a la fecha. 
\item 
Análisis de las disposiciones fiscales mexicanas que resulten aplicables contenidas en la 
Ley del Impuesto sobre la Renta (LISR), Ley del Impuesto al Valor Agregado (LIVA), 
Código Fiscal de la Federación (CFF), sus reglamentos, así como en las reglas de carácter 
general relacionadas. 
Nuestros comentarios respecto del cumplimiento razonable de las obligaciones fiscales 
que, en su caso, se hayan generado para el Cliente en México derivado de su 
participación en la estructura de referencia. 
\item 
Seguimiento, ya sea vía telefónica, correo electrónico y/o mediante reuniones que sean 
necesarias, para atender preguntas e inquietudes que resulten de lo anterior.
\end{enumerate}
% )))

\subsection*{\it Segunda Fase:} % (((
En la siguiente fase nuestros servicios podrían identificarse de la siguiente manera: 
\begin{enumerate}[I.]
\setcounter{enumi}{3}
\item 
Revisión y análisis de la información y documentación adicional proporcionada por el 
Cliente relativa a su patrimonio actual, incluyendo los planes de inversión que tiene 
pensado llevar a cabo y que involucran fundamentalmente instrumentos de renta 
variable, a través de instituciones en los Estados Unidos de América.  
\item 
Análisis de las disposiciones fiscales mexicanas que resulten aplicables contenidas en la 
LISR, LIVA, CFF, sus reglamentos, así como en las reglas de carácter general relacionadas; 
para determinar de manera general las implicaciones fiscales que se generarían en 
México con motivo de los rendimientos y flujos obtenidos a partir de dichas inversiones. 
Igualmente, análisis de las disposiciones fiscales americanas relevantes que resulten 
aplicables, así como de las disposiciones contenidas en el tratado México-Estados 
Unidos para evitar la doble tributación, de ser aplicable en el caso particular. 
Emitiríamos un documento con nuestros comentarios generales al respecto.  
\item 
Seguimiento, ya sea vía telefónica, correo electrónico y/o mediante reuniones que sean 
\end{enumerate}
% )))

\subsection*{\it Tercer Fase:} % (((
Por lo que respecta a la última fase, nuestros servicios podrían identificarse de la siguiente 
\begin{enumerate}[I.]
\setcounter{enumi}{6}
\item 
Revisión y análisis de la información y documentación adicional proporcionada por el 
Cliente relativa a la siguiente fase de sus planes de inversión y que comprenden 
inversiones de renta fija a través de una entidad ubicada en una jurisdicción diferente a 
Canadá y los Estados Unidos de América. 
\item 
Análisis de las disposiciones fiscales mexicanas que resulten aplicables contenidas en la 
LISR, LIVA, CFF, sus reglamentos, así como en las reglas de carácter general relacionadas; 
para determinar de manera general las implicaciones fiscales que se generarían en 
México con motivo de los rendimientos y flujos obtenidos a partir de dichas inversiones. 
Análisis de las disposiciones fiscales americanas relevantes que resulten aplicables, así 
como de las disposiciones contenidas en los tratados para evitar la doble tributación 
celebrados por México que sean aplicables en el caso particular. Emitiríamos un 
documento con nuestros comentarios generales al respecto.  
\item 
Seguimiento, ya sea vía telefónica, correo electrónico y/o mediante reuniones que sean 
necesarias, para atender preguntas e inquietudes que resulten de lo anterior. 
\end{enumerate}
 
% )))

Nuestros servicios comprenderían únicamente las actividades antes señaladas y no 
comprenderían la práctica de auditorías, ni la elaboración de dictámenes fiscales, ni trámites 
ante autoridad alguna americana o mexicana, ni la constitución de entidades jurídicas 
mexicanas, americanas o de otro país, ni la atención de requerimientos por parte de las 
autoridades fiscales americanas o mexicanas, ni la interposición o seguimiento de algún tipo de 
medio de defensa ante las autoridades fiscales o administrativas, mexicanas o americanas.  
 
Cabe mencionar que los servicios referidos correspondientes a los Estados Unidos serán 
prestados por un abogado con licencia para practicar en dicho país. Ello, derivado de las alianzas 

% )))

\section*{Honorarios.} % (((
Nuestros honorarios regularmente se fijan por proyecto, considerando el tiempo incurrido por 
el personal del Estudio que interviene en el desarrollo del trabajo, así como la complejidad del 
mismo, con base en las cuotas horarias vigentes, las cuales se actualizan anualmente. \\ 

No obstante lo antedicho, en el caso particular, con base en nuestra experiencia en trabajos 
similares, consideramos que el monto de nuestros honorarios por los servicios descritos en la 
primera fase ascendería a \$30,000 MXP.  \\ 

Ahora bien, por lo que se refiere a las actividades descritas en la segunda fase, estimamos que 
nuestros honorarios ascenderían a \$115,000 MXP. \\ 

Finalmente, por lo que se refiere a las actividades descritas en la tercera fase, estimamos que 
nuestros honorarios serían del órden de \$80,000 MXP. \\ 

Cabe señalar que el monto anterior ya contiene una consideración en nuestros honorarios 
ordinarios reflejando la intención de mantener una relación a largo plazo contigo y la 
colaboración en otros proyectos.

De ser aceptada la presente propuesta de servicios profesionales, nuestros honorarios serían 
facturados de manera separada por cada fase. Asimismo, los honorarios se facturarían en un 
50\% al momento de aceptación de la presente propuesta y el restante al momento de la 
terminación de los servicios comprendidos en cada fase. \\
 
Por otro lado, todas las cantidades anteriores deberán ser adicionadas con el impuesto al valor 
agregado correspondiente y no incluyen gastos notariales, pago de derechos u honorarios, 
gastos extraordinarios (out-of-pocket) que fuera necesario incurrir de tiempo en tiempo. De ser 
necesario incurrir en alguno de estos conceptos lo pondríamos primero a su consideración y 
aprovación.
% )))

\section*{Otros acuerdos.} % (((
En relación con lo anterior, el Estudio se obliga a prestar los servicios profesionales especificados 
en la presente carta. Cualquier cambio a los términos de la presente deberá obrar en una nueva 
carta modificatoria.  \\ 

El Cliente será responsable de proporcionar al Estudio todos los elementos necesarios 
(incluyendo información y documentación) de forma oportuna para que pueda llevarse a cabo 
los servicios profesionales descritos con anterioridad. \\ 

Los entregables proporcionados por el Estudio como consecuencia del desarrollo de su trabajo 
serán para uso único y exclusivo del Cliente y no podrán ser distribuidos a ningún tercero sin el 
consentimiento previo y por escrito del Estudio. \\ 

El Estudio proporcionará los servicios profesionales solicitados empleando por su exclusiva 
cuenta y responsabilidad el personal que a su juicio sea necesario o conveniente, acordando el 
Estudio y el Cliente que dicho personal no tendrá relación o nexo laboral alguno con el Cliente.  \\ 

La comunicación que se tendrá como consecuencia de los servicios profesionales solicitados, al 
menos, comprenderá el siguiente contacto del Cliente: \url{fjvaqueiro@augeconsultoria.mx}.  \\ 

Por medio de la firma de la presente carta propuesta, el Estudio y el Cliente aceptan los términos 
y condiciones de colaboración y contratación previstos con anterioridad. Dichos términos 
podrán modificarse o ajustarse conforme se requiera, por escrito suscrito de mutuo acuerdo por 
el Estudio y el Cliente. De no establecerse alguna otra disposición específica en el cuerpo de la 
presente carta, los términos y condiciones de la presente tendrán una vigencia de 40 días 
calendario contados a partir de su fecha de envío al Cliente hasta su aprobación por el mismo. \\ 

Finalmente, agradecemos la confianza depositada en el Estudio para la atención de su caso y la 
presentación de esta carta propuesta.
% )))

% === FIRMA === (((
\begin{center}
	Atentamente, \\ 
	\vspace{2cm}
	Juan Manuel López Durán. \\ 
	Socio
\end{center}
\textbf{Aceptado por:} \\ 
\hfill 
Nombre: 
\hfill 
Firma:
\hfill 
% )))

\end{document}
